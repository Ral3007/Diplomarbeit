\chapter{Verifikation}
\label{Verifikation}

Ziel der Diplomarbeit ist eine nutzerfreundliche Oberfläche zur Problemlösung in modularen Anlangen. Dieses Kapitel beantwortet, wie gut die Umsetzung geglückt ist.

\section{Validierung}
\subsection*{Vergleich mit Anforderung}
\begin{table}[htbp]
\centering
\begin{tabular}{l|l|l|l}
Anforderung & Erfüllt & Teilweise erfüllt & Nicht erfüllt \\
\hline
PI 1.1 & & x & \\
\hline
PI 1.1 & x & & \\
\hline

\end{tabular}
\end{table}

\subsection*{Eingliederung in Automatisierungsgrad}
Eingliederung in Automatisierung

\subsection*{Anforderungen an Assistenz}
Anforderungen an digitale Assistenz -> wie kann ich das messen?

Einsatz digitaler Assistenz -> Eingliedern, was kann die Assistenz alles?

\subsection*{Ergonomisch gute Gestaltung}
Ergonomisch gute Gestaltung

\subsection*{Individualisierbarkeit}
Ebene der Veränderung - Individualisierung


\section{Nutzerbefragung}


\subsection{Testmöglichkeiten}
Unterscheidung individuelle vs generalisierte Fragebögen
Andere Möglichkeiten des Testens: Beoabachten, etc...
Auswahl des Fragebogens -> was ist möglich?

\subsubsection*{System Usability Scale}
Der SUS (System Usability Scale) Fragebogen ist ein standardisierter Fragebogen zur Bewertung der System Usability. Er besteht aus zehn Fragen und fünf Stufen zwischen Stimme Gar nicht zu und Stimme voll zu. Für jede Frage werden abhängig von der Antwort zwischen 0 (Stimme gar nicht zu) und 4 (Stimme voll zu) Punkte verteilt und anschließend summiert. 

\subsubsection*{User Experience Questionnaire}

\subsubsection*{Computer System Usability Questionnaire}

\subsection{Der Fragebogen}

\subsection{Auswertung des Fragebogens}
Auswertung der Antworten

\section{Weiterer Use Case}

\section{Ausblick}
Hier oder ans Ende der Arbeit?
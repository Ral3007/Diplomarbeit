\chapter{Zusammenfassung und Ausblick}
\label{Zusammenfassung}
Das entwickelte Konzept für ein Assistenzsystem begleitet den Nutzer durch die verschiedenen Phasen des Problemlöseprozess. Es unterstützt bei der Problemidentifikation, indem es mit Meldungen, Warnungen und Alarmen auf ein Problem aufmerksam macht. Dabei wird unterschieden, wie zeitkritisch das Problem ist und welchem Bereich es zugeordnet werden kann. Es werden die auslösenden Bereiche Modul, KPI, Rezept und Service unterschieden. Entscheidet der Nutzer das Problem zu bearbeiten, geht der Problemlöseprozess in die Ziel- und Situationsanalyse über. In dieser Phase zeigt das Assistenzsystem an, welche Zusammenhänge zwischen dem auslösenden Bereich und dem Rest der modularen Anlage bestehen. Zusätzlich bekommt der Nutzer die Möglichkeit, Ziele für das Problem ein zu geben. Die Ziele orientieren sich an den Einflussgrößen in einem Unternehmen, die wichtig sind, um die Produktion aufrecht zu halten. Anhand dieser Ziele kann das Assistenzsystem nach Lösungen suchen. Damit ist die Phase Planerstellung erreicht. Dem Nutzer werden nun verschiedene Lösungsmöglichkeiten angezeigt. Jede Lösung zeigt an, welche Elemente (z. B. Services) der modularen Anlage sich verändern und xxxx \todo{was zeigt die Lösung an?}. Durch den permanenten Austausch zwischen Assistenz und Nutzer entsteht ein kollaborativer Problemlöseprozess.

Möchte der Nutzer den Problemlöseprozess unterbrechen oder tritt ein neues relevanteres Problem auf, wird der aktuelle Zustand gespeichert. Das Assistenzsystem sortiert die aktuellen Probleme anhand von verbleibender Zeit, Komplexität und zu erwartendem Arbeitsaufwand. So kann der Nutzer jederzeit zwischen Problemen wechseln und sehen, welches als nächstes gelöst werden sollte.

Das Experteninterview zeigt, dass die Ideen für das Assistenzsystem in die richte Richtung gehen. Der Prototyp wird in Hinblick auf einfache Bedienbarkeit, Übersichtlichkeit und xx sehr positiv wahrgenommen. \todo{ausformulieren}

\section*{Ausblick}
Diese Arbeit zeigt, dass viele Faktoren den Entwurf eines Assistenzsystems für modulare Anlagen beeinflussen. Entsprechend viele Bereiche wurden in dieser Arbeit nicht betrachtet oder nur angeschnitten. Der Prototyp und auch Teile des Konzepts bauen auf einem konkreten Use Case auf. Das Konzept betrachtet, welche groben Zusammenhänge zwischen Problemauslöser und Gesamtanlage bestehen. Offen bleibt, welche Probleme im Detail entstehen bzw. entstehen könnten und wie sich diese konkret unterschieden lassen. Eine Idee ist, zunächst zwischen anlagen- und modulspezifischen Problem zu unterscheiden. Im Zuge dessen ist auch zu untersuchen, welche Daten vom Modul an die Assistenz und welche Daten von der Assistenz an die Interaktionsplattform übertragen werden müssen. Bei der Datenübertragung ist sowohl die Problembeschreibung als auch die Lösungsfindung zu berücksichtigen. Es ist noch nicht geklärt, welche Daten vom MTP bereitgestellt werden müssen, damit die Assistenz Lösungen finden kann. Dafür ist auch zu untersuchen, welche Möglichkeiten es zur Lösungsfindung überhaupt geben kann. Basieren Lösungen auf schon vorher bearbeitenden Problemen? Hat der Nutzer über eine virtuelle modulare Anlagen die Möglichkeit, vorab Lösungen zu testen? Kann die Assistenz die Lösung von ähnlichen Probleme mit der virtuellen Anlage testen, um vorab zu bestimmen, ob es eine sinnvolle Lösung ist? Diese Fragen müssen für die Anwendung der Nutzeroberfläche noch beantwortet werden.
---------------------------------------------

Welche Informationen sind für welche Ziele relevant?

Wie unterscheiden sich Meldungen, Warnungen, Alarme hinsichtlich Zeit, Komplexität? Kann man dort schon Probleme unterscheiden? Welche Informationen müssen übergeben werden?



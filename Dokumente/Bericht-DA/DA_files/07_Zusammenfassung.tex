\chapter{Fazit}
\label{Zusammenfassung}
Die Modularisierungskonzepte für die Prozessindustrie erhöhen die Flexibilität bei der Entwicklung von Anlagen. Sie stellen den Nutzer vor die Herausforderung, Probleme nicht mehr auf Grundlage von umfangreicher Erfahrung lösen zu können \cite{Muller2018}.

Für eine geeignete Unterstützung wurde identifiziert, was bei einem Assistenzsystem für modulare Anlagen zu berücksichtigen ist. Darauf aufbauend entstand eine Nutzeroberfläche, die Probleme und Lösungen darstellt. Die Nutzeroberfläche dient zur Kommunikation zwischen Mensch und Assistenz. Die Assistenz analysiert im Hintergrund die vorhandenen Informationen und teilt sie dem Nutzer über die Interaktionsplattform mit. Dadurch entsteht ein kollaborativer Problemlöseprozess. Anhand eines Use Case wurde ein Prototyp entwickelt, der auf den entworfenen Konzepten aufbaut. Mit diesem konnte eine Umfrage unter Experten durchgeführt werden. Die Auswertung der Umfrage zeigt, dass die Nutzeroberfläche, bei Erweiterung um ein intelligentes System, Anwendung findet.

\section{Zusammenfassung}
Assistenzsysteme können den Menschen bei vielen Dingen unterstützen. Die Möglichkeiten reichen von einfachen Hinweisen bis hin zum automatischem Ausführen von Aufgaben \cite{Wandke2005}. Jede Art von Unterstützung fordert eine angemessene Gestaltung. Für den Nutzer ist es wichtig, dass die Verwendung des Assistenzsystems nicht zu Frustration sondern zu Freude, Spaß und Stolz führt \cite{Hassenzahl2008}. Relevant ist das insbesondere mit Blick auf einen komplexen Problemlöseprozess. Dieser wird sowohl durch die Emotionen des Menschen als auch die Art und Menge der bereitgestellten Informationen beeinflusst. 

Bei der Entwicklung eines Assistenzsystems für modulare Anlagen sind einerseits die Informationen, die vom Modul bereit gestellt werden, andererseits die notwendigen Informationen für ein produzierendes Unternehmen zu berücksichtigen. Die bereits entwickelte Prozessführungsebene zeigt nur einen Teil der vorhandenen Informationen an und unterstützt den Nutzer bei auftretenden Problemen nicht. Informationen, die vom Modul bereit gestellt aber nicht angezeigt werden, sind unter anderem Meldungen, Warnung und Alarme, Serviceabhängigkeiten und das zugehörige Equipment der Services. Für eine erfolgreiche Problemlösung müssen die zu erreichenden Ziele klar sein. Diese orientieren sich an den Einflussgrößen in einem produzierenden Unternehmen. Ein Unternehmen besteht aus verschiedene Ebenen. Auf der Ebene der Produktionsplanung ist unter anderem die Kapazitätsoptimierung relevant, auf die in dieser Arbeit der Schwerpunkt gelegt wurde. Um den Anlagenbediener bei seiner Aufgabe zu unterstützen, findet ein mittlerer Grad an Automatisierung Anwendung. Dies wird laut Sauer und Chavaillaz \cite{Sauer2018} von Anlagenbedienern bevorzugt und generiert die meisten Lösungsvarianten \cite{Miller2005}.

Die entwickelte Nutzeroberfläche begleitet den Nutzer durch die Phasen des Problemlöseprozesses. In Phase eins, der Problemidentifikation, wird auf Meldungen, Warnungen und Alarme aufmerksam gemacht, die ein Problem verursachen. Ein Problem entsteht erst, wenn die Meldungen, Warnungen und Alarme nicht mit den Zielen des Unternehmens vereinbar sind. Das Assistenzsystem identifiziert, welchem Bereich das Problem zugeordnet ist. Es werden die auslösenden Bereiche Modul, KPI, Rezept und Service unterschieden. Entscheidet der Nutzer das Problem zu bearbeiten, geht der Problemlöseprozess in Phase zwei, die Ziel- und Situationsanalyse, über. In dieser hebt das Assistenzsystem hervor, welche Zusammenhänge zwischen Problem und dem Rest der modularen Anlage bestehen. Zusätzlich wird der Nutzer bei der Parametrierung der Ziele unterstützt. Er hat die Möglichkeit, die Parameter der vorgeschlagenen Ziele zu verändern, Ziele hinzuzufügen oder zu löschen. Wenn der Nutzer seine Eingaben bestätigt, folgt Phase drei, die Planerstellung. Dabei sucht das Assistenzsystem nach geeigneten Lösungen, die sich am Problem und den gesetzten Zielen orientieren. Es wird dem Nutzer zum einen angezeigt, welche Elemente (z. B. Services) sich durch Anwendung der Lösung in der modularen Anlage verändern. Zum anderen können in einer tabellarischen Übersicht die Lösungen anhand ihrer Parameter, wie entstehende Kosten und voraussichtlicher Zeitaufwand, verglichen werden. Diese Angaben sollen eine Entscheidung erleichtern.

Die Umfrage unter den Experten ergab, dass die Entscheidung zwar erleichtert wird, sie jedoch immer noch schwierig ist. Sie wünschen sich einheitlich, dass zu der notwendigen Anzahl an Mitarbeitern auch angezeigt wird, welche Kosten ein Mitarbeiter verursacht. Durch Angabe der Gesamtkosten erhoffen sich die Experten eine bessere Vergleichbarkeit der einzelnen Lösungen. Insgesamt wurde der Automatisierungsgrade des Assistenten positiv bewertet. Die Auswahl an Zielen und Lösungen ist, laut der Experten, angemessen, da diese den Nutzer gut leiten. Aufgrund des schlichten Aufbaus, der Übersichtlichkeit und der einfachen Bedienung wird die Nutzeroberfläche von den Experten empfohlen. Für eine konkrete Anwendung der Nutzeroberfläche wird zudem gewünscht, den Nutzer auch bei der Durchführung der Lösung zu unterstützen.

\section{Ausblick}
Diese Arbeit zeigt, dass der Entwurf einen Assistenzsystems für modulare Anlagen vielschichtig ist und von vielen Faktoren beeinflusst wird. Für eine erfolgreiche Anwendung der Nutzeroberfläche sind noch einige Bereiche zu untersuchen. Der Prototyp und auch Teile des Konzepts bauen auf einem konkreten Use Case auf. Das Konzept betrachtet die bestehenden Zusammenhänge zwischen Problemauslöser und Gesamtanlage. Für eine ideale Anpassung an Probleme bleibt zu überprüfen, ob die groben Bereiche Modul, KPI, Rezept und Service ausreichend sind, um den Nutzer ideal zu unterstützen. Mit diesen Informationen kann die Nutzerplattform weiter entwickelt und getestet werden. Bis jetzt ist noch nicht eindeutig, welche Probleme in modularen Anlagen im Detail auftreten.  Eine Möglichkeit wäre zunächst zwischen anlagen- und modulspezifischen Problemen zu unterscheiden. Um das Problem und die Lösungen geeignet darzustellen, benötigt die Nutzeroberfläche eine ganze Reihe an Informationen. Diese reichen vom Problembereich, über die Ziele bis hin zu den Zusammenhängen. Es bleibt zu untersuchen, wie diese Menge an Informationen vom Modul an die Assistenz und von der Assistenz an die Interaktionsplattform übertragen werden können. Bei der Datenübertragung ist sowohl die Problembeschreibung als auch die Lösungsfindung zu berücksichtigen.

Damit die Assistenz Lösungen finden kann, ist zu klären, welche Daten vom MTP bereitgestellt werden müssen. Dafür sind auch anwendbare Möglichkeiten zur Lösungsfindung zu prüfen. Wenn Lösungen auf vorher bearbeiteten Problemen aufbauen und dies dem Nutzer mitgeteilt wird, könnte er sich leichter für eine Lösung entscheiden und Erfahrungswissen aufbauen. Die entwickelte Nutzeroberfläche kann für eine solche Untersuchung, bei Ergänzung der entsprechenden Information, verwendet werden. Hat der Nutzer und/oder die Assistenz die Möglichkeit, Lösungen vorab an einer virtuell modularen Anlage zu testen, könnten dadurch Fehlentscheidungen reduziert werden. Durch umfassende Analyse der thematisierten Faktoren kann aus der Nutzeroberfläche ein intelligentes Assistenzsystem werden, das dann in modularen Anlagen verwendbar ist.
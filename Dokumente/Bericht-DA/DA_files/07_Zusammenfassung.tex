\chapter{Fazit}
\label{Zusammenfassung}
Die Modularisierungskonzepte für die Prozessindustrie erhöhen die Flexibilität bei der Entwicklung von Anlagen. Sie stellen den Nutzer vor die Herausforderung, Probleme nicht mehr auf Grundlage von umfangreicher Erfahrung lösen zu können. Für eine geeignete Unterstützung wurde identifiziert, was bei einem Assistenzsystem für modulare Anlagen zu berücksichtigen ist. Darauf aufbauend entstand eine Nutzeroberfläche, die Probleme und Lösungen darstellt. Die Nutzeroberfläche dient zur Kommunikation zwischen Mensch und Assistenz. Die Assistenz analysiert im Hintergrund die vorhandenen Informationen und teilt sie dem Nutzer über die Interaktionsplattform mit. Dadurch entsteht ein kollaborativer Problemlöseprozess. Anhand eines Use Case wurde ein Prototyp entwickelt, der auf den entworfenen Konzepten aufbaut. Mit diesem konnte eine Umfrage unter Experten durchgeführt werden. Die Auswertung der Umfrage zeigt, dass die Nutzeroberfläche, bei Erweiterung um ein intelligentes System, Anwendung findet.

\section{Zusammenfassung}
Assistenzsysteme können den Menschen bei vielen Dingen unterstützen. Die Möglichkeiten reichen von einfachen Hinweisen bis hin zum automatischem Ausführen von Aufgaben. Jede Art von Unterstützung fordert eine angemessene Gestaltung. Für den Nutzer ist es wichtig, dass die Verwendung des Assistenzsystems nicht zu Frustration sondern zu Freude, Spaß und Stolz führt. Relevant ist das insbesondere mit Blick auf einen komplexen Problemlöseprozess. Dieser wird sowohl durch die Emotionen des Menschen als auch die Art und Menge der Informationen beeinflusst. 

Bei der Entwicklung eines Assistenzsystems für modulare Anlagen sind einerseits die Informationen, die vom Modul bereit gestellt werden, andererseits die notwendigen Informationen für ein produzierendes Unternehmen zu berücksichtigen. Die bereits entwickelte Prozessführungsebene zeigt nur einen Teil der vorhandenen Informationen an und unterstützt den Nutzer bei auftretenden Problemen nicht. Informationen, die vom Modul bereit gestellt aber nicht angezeigt werden, sind unter anderem Meldungen, Warnung und Alarme, Serviceabhängigkeiten und das zugehörige Equipment der Services. Für eine erfolgreiche Problemlösung müssen die zu erreichenden Ziele klar sein. Diese orientieren sich an den Einflussgrößen in einem produzierenden Unternehmen. Ein Unternehmen besteht aus verschiedene Ebenen \todo{umformulieren}. Auf der Ebene der Produktionsplanung ist unter anderem die Kapazitätsoptimierung relevant, auf die in dieser Arbeit der Schwerpunkt gelegt wurde. Um dem Anlagenbediener bei dieser Aufgabe zu unterstützen ist ein sinnvoller Grad der Automatisierung zu wählen. \todo{Nutzer??} 

Die entwickelte Nutzeroberfläche begleitet den Nutzer durch die Phasen des Problemlöseprozess. In Phase eins, der Problemidentifikation, wird auf Meldungen, Warnungen und Alarme aufmerksam gemacht, die ein Problem verursachen. Ein Problem entsteht erst, wenn die Meldungen, Warnungen und Alarme nicht mit den Zielen des Unternehmens vereinbar sind. Das Assistenzsystem identifiziert, welchem Bereich das Problem zugeordnet ist. Es werden die auslösenden Bereiche Modul, KPI, Rezept und Service unterschieden. Entscheidet der Nutzer das Problem zu bearbeiten, geht der Problemlöseprozess in Phase zwei, die Ziel- und Situationsanalyse, über. In dieser hebt das Assistenzsystem hervor, welche Zusammenhänge zwischen Problem und dem Rest der modularen Anlage bestehen. Zusätzlich wird der Nutzer bei der Parametrierung der Ziele unterstützt. \todo{Was macht die Assistenz?} In Phase drei, der Planerstellung, sucht das Assistenzsystem nach geeigneten Lösungen. Diese orientieren sich am Problem und den gesetzten Zielen. Es wird dem Nutzer zum einen angezeigt, welche Elemente (z. B. Services) sich durch Anwendung der Lösung in der modularen Anlage verändern. Zum anderen können in einer tabellarischen Übersicht die Lösungen anhand ihrer Parameter, wie entstehende Kosten und voraussichtlicher Zeitaufwand, verglichen werden. Diese Angaben sollen eine Entscheidung erleichtern.

Die Umfrage unter den Experten ergab, dass die Entscheidung zwar erleichtert wird, sie jedoch immer noch schwierig ist. Sie wünschen sich einheitlich, dass zu der notwendigen Anzahl an Mitarbeitern auch angezeigt wird, welche Kosten ein Mitarbeiter verursacht. Durch diese Angabe erhoffen sich die Experten eine bessere Vergleichbarkeit der einzelnen Lösungen. 



%Das entwickelte Konzept für ein Assistenzsystem begleitet den Nutzer durch die verschiedenen Phasen des Problemlöseprozess. Es unterstützt bei der Problemidentifikation, indem es mit Meldungen, Warnungen und Alarmen auf ein Problem aufmerksam macht. Dabei wird unterschieden, wie zeitkritisch das Problem ist und welchem Bereich es zugeordnet werden kann. Es werden die auslösenden Bereiche Modul, KPI, Rezept und Service unterschieden. Entscheidet der Nutzer das Problem zu bearbeiten, geht der Problemlöseprozess in die Ziel- und Situationsanalyse über. In dieser Phase zeigt das Assistenzsystem an, welche Zusammenhänge zwischen dem auslösenden Bereich und dem Rest der modularen Anlage bestehen. Zusätzlich bekommt der Nutzer die Möglichkeit, Ziele für das Problem ein zu geben. Die Ziele orientieren sich an den Einflussgrößen in einem Unternehmen, die wichtig sind, um die Produktion aufrecht zu halten. Anhand dieser Ziele kann das Assistenzsystem nach Lösungen suchen. Damit ist die Phase Planerstellung erreicht. Dem Nutzer werden nun verschiedene Lösungsmöglichkeiten angezeigt. Jede Lösung zeigt an, welche Elemente (z. B. Services) der modularen Anlage sich verändern und xxxx \todo{was zeigt die Lösung an?}. Durch den permanenten Austausch zwischen Assistenz und Nutzer entsteht ein kollaborativer Problemlöseprozess.

%Das Experteninterview zeigt, dass die Ideen für das Assistenzsystem in die richte Richtung gehen. Der Prototyp wird in Hinblick auf einfache Bedienbarkeit, Übersichtlichkeit und xx sehr positiv wahrgenommen. \todo{ausformulieren}

\section{Ausblick}
Diese Arbeit zeigt, dass viele Faktoren den Entwurf eines Assistenzsystems für modulare Anlagen beeinflussen. Für eine erfolgreiche Umsetzung sind noch entsprechend viele Bereich zu untersuchen. Der Prototyp und auch Teile des Konzepts bauen auf einem konkreten Use Case auf. Das Konzept betrachtet, welche groben Zusammenhänge zwischen Problemauslöser und Gesamtanlage bestehen. Damit sich die Interaktionsplattform ideal anpassen kann, muss noch erforscht werden, welche Probleme im Detail entstehen bzw. entstehen könnten und wie sich diese konkret unterscheiden lassen. Eine Idee ist, zunächst zwischen anlagen- und modulspezifischen Problemen zu unterscheiden. Im Zuge dessen ist auch zu untersuchen, welche Daten vom Modul an die Assistenz und welche Daten von der Assistenz an die Interaktionsplattform übertragen werden müssen. Bei der Datenübertragung ist sowohl die Problembeschreibung als auch die Lösungsfindung zu berücksichtigen. Es ist noch nicht geklärt, welche Daten vom MTP bereitgestellt werden müssen, damit die Assistenz Lösungen finden und den Nutzer ideal unterstützen kann. Dafür ist auch zu untersuchen, welche Möglichkeiten zur Lösungsfindung anwendbar sind. Es stellen sich folgende Fragen: Basieren Lösungen auf schon vorher bearbeitenden Problemen? Hat der Nutzer über eine virtuelle modulare Anlagen die Möglichkeit, vorab Lösungen zu testen? Kann die Assistenz die Lösung von ähnlichen Probleme mit der virtuellen Anlage testen, um vorab zu bestimmen, ob es eine sinnvolle Lösung ist? Diese Fragen müssen für die Anwendung der Nutzeroberfläche noch beantwortet werden.
---------------------------------------------

Welche Informationen sind für welche Ziele relevant?

Wie unterscheiden sich Meldungen, Warnungen, Alarme hinsichtlich Zeit, Komplexität? Kann man dort schon Probleme unterscheiden? Welche Informationen müssen übergeben werden?



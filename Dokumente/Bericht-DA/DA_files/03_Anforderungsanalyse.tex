%Anfoderungsanalyse

%%%%%%%%%%%%%%%%%%%%%%%%
\chapter{Analyse}
\label{sec:Anforderungsanalyse}
%%%%%%%%%%%%%%%%%%%%

\section{Informationsbedarf}
Der Informationsbedarf orientiert sich maßgeblich an den Aufgaben und dem Nutzer.

Was soll unterstützt werden?

\subsection{Aufgabenbereiche}
Zunächst einmal wird der Lebenszyklus einer modulare Anlage in unterschiedliche Phasen eingeteilt. Sie besteht aus Planung, Errichtung, Betrieb und Demontage. \cite{} Ich möchte mich in dieser Arbeit auf die Probleme während des Betriebs der modularen Anlage beschränken.

Probleme können in modularen Anlagen auf verschiedenen Ebenen entstehen. Zum einen auf direkter Modulebene. So stellt das Modul beispielsweise entsprechende Alarme zur Verfügung. Dann können Probleme auch in Zusammenhang mit verschiedenen Modulen entstehen. Folgende Informationen stellt das Modul zur Verfügung:

\subsection{Nutzer}


\section{Informationsanpassung}
Die Individualisierung von Software bietet die Möglichkeit eine Vielzahl von Nutzer und Aufgaben zu unterstützen. Individualisierung dient der Modifizierung von Interaktionen und Informationsdarstellungen, um den Fähigkeiten und Bedürfnissen jedes Benutzers gerecht zu werden. Zudem kann sich das System auch entsprechend der zu lösenden Aufgaben anpassen.

\subsection{Individualisierung für den Menschen}

\subsection{Anpassung an die Aufgabe}
Da die entstehenden Probleme sehr unterschiedlich sein können ist eine entsprechende Anpassung an die aktuelle Situation und die zu bearbeitende Aufgabe wichtig.

Wie schon in Abschnitt x beschrieben lassen sich Probleme anhand verschiedener Aspekte unterscheiden. So ist zum Beispiel der Zeitdruck ein wichtiger Aspekt. Bei zeitkritischen Problemen muss möglichst schnell eine gute Lösung gefunden werden. Ist das Problem nicht zeitkritisch so können in Ruhe alle zur Verfügung stehenden Informationen in den Problemlöseprozess mit einbezogen werden. So kann bei einem zeitkritischen Problem ein höherer Automatisierungsgrad gefordert sein. Um dennoch dem Menschen seine Kompetenzen nicht abzusprechen ist es möglich bei Problemen, die eher langfristig sind und die eine höhere kognitive Aktivität erfordern, eine geringere Autonomiestufe anzuwenden.

Des weiteren unterscheidet sich maßgeblich, welche Informationen zur Verfügung gestellt werden müssen. Sendet ein Modul beispielsweise eine Warnung, dass der Füllstand den Grenzwert bald überschreitet so ist möglicherweise ein Hinweis notwendig, wie lange es noch dauert, bis der Behälter überläuft und was mögliche Konsequenzen sind. 

\section{Interaktionsmechaniken}
Im Kontext dieser Arbeit wird das Problemlösen betrachtet. Problemlösen heißt in diesem Fall, dass beispielsweise die Ursache für eine Störung ausfindig gemacht werden muss. Die Behebung der Ursache, z.B. durch eine Reparatur, wird an dieser Stelle ausgeklammert. Im Stand der Technik sind verschiedene Interaktionsmechaniken beschrieben. Um diese geeignet bewerten zu können ist zunächst eine Begutachtung des Arbeitsumfelds und der Aufgaben notwendig. 

\section{Anforderungen an das Assistenzsystem}

Unterstützung des Problemlöseprozess: Problem identifizieren. Ziel festlegen.

Probleme sortieren

Es muss eine Einschätzung erfolgen können, wie zeitkritisch das Problem ist


%Anfoderungsanalyse

%%%%%%%%%%%%%%%%%%%%%%%%
\chapter{Analyse}
\label{sec:Anforderungsanalyse}
%%%%%%%%%%%%%%%%%%%%

\section{Informationsbedarf}

\section{Informationsanpassung}

Die Individualisierung von Software bietet die Möglichkeit eine Vielzahl von Nutzer und Aufgaben zu unterstützen. Individualisierung dient der Modifizierung von Interaktionen und Informationsdarstellungen, um die Fähigkeiten und Bedürfnisses jedes Benutzers gerecht zu werden.

\section{Interaktionsmechaniken}
Im Kontext dieser Arbeit wird das Problemlösen betrachtet. Problemlösen heißt in diesem Fall, dass beispielsweise die Ursache für eine Störung ausfindig gemacht werden muss. Die Behebung der Ursache, z.B. durch eine Reparatur, wird an dieser Stelle ausgeklammert. Im Stand der Technik sind verschiedene Interaktionsmechaniken beschrieben. Um diese geeignet bewerten zu können ist zunächst eine Begutachtung des Arbeitsumfelds und der Aufgaben notwendig.


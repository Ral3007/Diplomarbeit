%Beschreibung der Problematik

%%%%%%%%%%%%%%%%%%%%%%%%%%%%%%%%%%%%%%%%%%%%%%%%%%%%%%%%%
%%%%%%%%%%%%%%%%%%%%%%%%%%%%%%%%%%%%%%%%%%%%%%%%%%%%%%%%%
\chapter{Einleitung}
\label{sec:Einleitung}
%%%%%%%%%%%%%%%%%%%%%%%%%%%%%%%%%%%%%%%%%%%%%%%%%%%%%%%%%

Auf Grundlage von immer kürzeren Produkteinführungszeiten wurden Modularisierungskonzepte für die Prozessindustrie entworfen. Dies bieten eine höhere Flexibilität, haben allerdings auch Auswirkungen auf das Engineering und den Betrieb der Anlage \cite{Obst2013}. Dies hat auch Auswirkungen auf die Störungsdiagnose. Selbst im herkömmlichen Anlagenbau werden die Mitarbeiter bei Problemen immer wieder aufs neue gefordert. So schrieb Bainbridge \cite{Bainbridge1983} schon 1983, dass die Automatisierung weitreichende Auswirkungen auf die Arbeit des Operators hat. So ist zwar die Anlage automatisiert, bei kritischen Situationen ist jedoch der Mensch gefordert. Dieser trifft seine Entscheidung anhand von Beobachtung und Erfahrungen. Da die Komplexität der Verfahren zur Produktion zunimmt ist es schwierig bei auftretenden Störungen alle Faktoren zu kennen und zu überblicken.

Assistenzsysteme können hier eine geeignete Unterstützung bieten \cite{Dalgleish2007} \todo{Zitat korrigieren (Teil eines Buchs)}. Dabei ist zu beachten, dass der Mensch nicht als Lückenbüßer verwendet wird, der alle Aufgaben übernehmen muss mit denen das Automatisierungssystem überfordert ist. Die Kompetenzen des Menschen sind zu würdigen und mit zusätzlichen Informationen aus dem Prozess zu ergänzen. \cite{Weisner2018}




Durch voranschreiten der Automatisierung in der Prozessführung sind Menschen vor allem in kritischen Situationen für Entscheidungen verantwortlich.

%Beschreibung der Problematik

%%%%%%%%%%%%%%%%%%%%%%%%%%%%%%%%%%%%%%%%%%%%%%%%%%%%%%%%%
%%%%%%%%%%%%%%%%%%%%%%%%%%%%%%%%%%%%%%%%%%%%%%%%%%%%%%%%%
\chapter{Einleitung}
\label{sec:Einleitung}
%%%%%%%%%%%%%%%%%%%%%%%%%%%%%%%%%%%%%%%%%%%%%%%%%%%%%%%%%

Durch voranschreiten der Automatisierung in der Prozessführung sind Operator vor allem in kritischen Situationen für Entscheidungen verantwortlich \cite{Bainbridge1983}. Der Mensch trifft seine Entscheidungen anhand von Beobachtungen und Erfahrungen. Da die Komplexität der Verfahren zur Produktion zunimmt ist es schwierig bei auftretenden Störungen alle Faktoren zu kennen und zu überblicken. 

Neben der Automatisierung verändern auch die entwickelten Modularisierungskonzepte für die Prozessindustrie die Aufgaben beim Betrieb der Anlage. Bei der Modularisierung besteht die Prozessanlage aus ein oder mehr Modulen, die eine verfahrenstechnische Funktion erfüllen und mittels Services gesteuert werden. \glqq Um dem Bedien- und Wartungspersonal Eingriffsmöglichkeiten zu geben, muss der Bezug zwischen örtlicher Kennzeichnung, innerhalb des Moduls und der Kennzeichnung im übergeordneten Automatisierungssystem bekannt gemacht werden.\grqq \ \cite{Obst2013} 

Assistenzsysteme können hier eine geeignete Unterstützung bieten \cite{Dalgleish2007} \todo{Zitat korrigieren (Teil eines Buchs)}. Dabei ist zu beachten, dass der Mensch nicht als Lückenbüßer verwendet wird, der alle Aufgaben übernehmen muss mit denen das Automatisierungssystem überfordert ist. Die Kompetenzen des Menschen sind zu würdigen und mit zusätzlichen Informationen aus dem Prozess zu ergänzen. \cite{Weisner2018}

\todo{Unterschied Problem, Störung, Fehler}


%!TEX root = ../example.tex
%*******************************************************************************
% * Copyright (c) 2006-2013 
% * Institute of Automation, Dresden University of Technology
% * 
% * All rights reserved. This program and the accompanying materials
% * are made available under the terms of the Eclipse Public License v1.0 
% * which accompanies this distribution, and is available at
% * http://www.eclipse.org/legal/epl-v10.html
% * 
% * Contributors:
% *   Institute of Automation - TU Dresden, Germany 
% *      - initial API and implementation
% ******************************************************************************/

% Dieser Befehl setzt das Nomenklaturverzeichnis mit angegebener Breite für die erste Spalte
\printnomenclature  [3 cm]

% Falls das Abkürzungs- und Symbolverzeichnis ohne Nutzung des nomencl-Pakets 
% von Ihnen selber erstellt werden soll (z.B. in einer Tabelle), ersetzen Sie 
% einfach den /printnomenclature - Befehl durch eigenen Quellcode.

% In Ihrem Dokument können Sie dann bei jeder Verwendung einer Abkürzung oder 
% eines neuen Symbols diese/dieses kurz über den \nomenclature Befehl erklären 
% und dem Abkürzungsverzeichnis zur Verfügung stellen.

% Ein Beispiel für ein Symbol ("yx"):
%\nomenclature[yx ]{$\sigma$}{Standardabweichung}

% Ein Beispiel für eine Abkürzung ("ba"):
\nomenclature[ba ]{z. B.}{zum Beispiel}
\nomenclature[ba ]{PFE}{Prozessführungsebene}
\nomenclature[ba ]{MTP}{Modul Type Package}
\nomenclature[ba ]{bspw.}{beispielsweise}
\nomenclature[ba ]{UX}{User Experience}
\nomenclature[ba ]{vgl.}{vergleiche}
\nomenclature[ba ]{HMI}{Human Machine Interface}
\nomenclature[ba ]{SUS}{System Usability Scale}

\section*{Glossar}
\vspace{-0.3 cm}
\hspace{-0.3 cm}
\begin{tabular}{p{4cm}p{9cm}}
Fehler & Ein Fehler ist eine nicht zulässige Abweichung. \\
Modul & Ein Modul erfüllt eine verfahrenstechnische Grundfunktion und ist gleichbedeutend mit Process Equipment Assembly (PEA). \\
Operator & Ein Mittel, das zum Erreichen eines Ziels verwendet werden kann. \\
Problem & Ein Problem tritt auf, wenn ein definiertes Ziel nicht ohne Weiteres erreicht werden kann. \\
Prozessführungsebene & Die Prozessführungsebene stellt die Konfiguration der Module dar. xxx \\
Störung & Eine Störung entsteht, wenn eine Funktion nicht ausgeführt werden kann. \\
Ziel & Im Zusammenhang mit dem Problemlöseprozess sind Ziele Randbedingungen, die sich an den Einflussgrößen in einem Unternehmen orientieren. \\
\end{tabular}

%!TEX root = ../example.tex
%*******************************************************************************
% * Copyright (c) 2006-2013 
% * Institute of Automation, Dresden University of Technology
% * 
% * All rights reserved. This program and the accompanying materials
% * are made available under the terms of the Eclipse Public License v1.0 
% * which accompanies this distribution, and is available at
% * http://www.eclipse.org/legal/epl-v10.html
% * 
% * Contributors:
% *   Institute of Automation - TU Dresden, Germany 
% *      - initial API and implementation
% ******************************************************************************/

% Dieser Befehl setzt das Nomenklaturverzeichnis mit angegebener Breite für die erste Spalte
\printnomenclature[4 cm]

% Falls das Abkürzungs- und Symbolverzeichnis ohne Nutzung des nomencl-Pakets 
% von Ihnen selber erstellt werden soll (z.B. in einer Tabelle), ersetzen Sie 
% einfach den /printnomenclature - Befehl durch eigenen Quellcode.

% In Ihrem Dokument können Sie dann bei jeder Verwendung einer Abkürzung oder 
% eines neuen Symbols diese/dieses kurz über den \nomenclature Befehl erklären 
% und dem Abkürzungsverzeichnis zur Verfügung stellen.

% Ein Beispiel für ein Symbol ("yx"):
%\nomenclature[yx ]{$\sigma$}{Standardabweichung}

% Ein Beispiel für eine Abkürzung ("ba"):
\nomenclature[ba ]{z. B.}{zum Beispiel}
\nomenclature[ba ]{PFE}{Prozessführungsebene}
\nomenclature[ba ]{MTP}{Module Type Package}
\nomenclature[ba ]{bspw.}{beispielsweise}
\nomenclature[ba ]{UX}{User Experience}
\nomenclature[ba ]{vgl.}{vergleiche}

\nomenclature[gl ]{Problem}{Ein Problem}
\nomenclature[gl ]{Störung}{Eine Störung}
\nomenclature[gl ]{Fehler}{Ein Fehler ist}
\nomenclature[gl ]{Modul}{Ein Modul erfüllt eine verfahrenstechnische Grundfunktion und ist gleichbedeutend mit Process Equipment Assembly (PEA)}
\nomenclature[gl ]{Prozessführungsebene}{xxx}
\nomenclature[gl ]{Process Orchestration Layer}{Das Process Orchestration Layer erweitert den Funktionsumfang einer Prozessführungsebene um....}
\nomenclature[gl ]{Service}{= Dienst}

%Beschreibung der Stand der Technik

%%%%%%%%%%%%%%%%%%%%%%%%
\chapter{Stand der Technik}
\label{sec:StandDerTechnik}

%%%%%%%%%%%%%%%%%%%%%%%%

\section{Modulare Anlagen}
Aufgrund immer kürzerer Produkteinführungszeiten werden Modularisierungskonzepte entwickelt. Die Modularisierung ermöglicht eine höhere Flexibilität und beschleunigt Konzeption, Engineering, Aufbau und Inbetriebnahme der Anlage \cite{Urbas2012}. Ein Modul ist eine geschlossene funktionale Einheit und stellt eine verfahrenstechnische Grundfunktion als Dienst der Prozessführungsebene (PFE) zur Verfügung. Die Grundfunktionalitäten der PFE müssen unterstützt werden \cite{Bernshausen2016}
\begin{itemize}
\item \textbf{Mensch-Maschine-Schnittstelle:} Übertragung der Daten zur Anzeige und Bedienung
\item \textbf{Steuern und Überwachen:} Übertragung der internen Zustände des Moduls
\end{itemize}

In der Namur-Empfehlung NE 148 \cite{NAMURArbeitskreis1.122013} ist beschrieben, welche Daten an das übergeordnete Automatisierungssystem übertragen werden und welche dem Modullieferanten zur Wartungsunterstützung zur Verfügung stehen. Die Daten für das übergeordnete Automatisierungssystemen umfassen unter anderem die Verriegelungs-, Steuerungs- und Reglungsstruktur, die Prozess- und Sollwerte sowie den Status des Moduls / der Services. Für die Wartungsunterstüzungen werden nur hersteller- und modulspezifische und keine prozessspezifischen Daten übertragen.


\section{Problemlösen}
Die Gesellschaft geht davon aus, dass Probleme selbstverständlich existieren. Probleme entstehen allerdings erst, wenn eine konkrete Zielsetzung vorhanden ist, die sich nicht durch Routine erreichen lässt. Ohne Handlungsziele gäbe es keine Probleme.

Liegt ein Problem vor so könnte der Problemlöseprozess sehr einfach sein, indem der Ausgangszustand erkannt, der Zielzustand festgelegt und die Operatoren gefunden werden. Allerdings haben alle diese Aspekte Eigenschaften, die den Prozess erschweren. So kann der Ausgangszustand nicht immer klar definiert sein und es muss eindeutig sein, welche Voraussetzungen als erfüllt angenommen werden können. Bei einem unklaren Ausgangszustand lässt sich auch der Zielzustand nicht eindeutig beschreiben. Bei Betrachtung der Operatoren, die notwendig sind, um einen Ausgangszustand in einen Zielzustand zu überführen, fällt auf, dass diese mit dem Ziel zusammen hängen. Entweder wird der Zielzustand betrachtet und nach geeigneten Operatoren gesucht, die unter Umständen nicht vorhanden sind. Oder es sind bestimmte Operatoren vorhanden und es wird davon ausgehend das bestmöglichste Ziel bestimmt.

Wann gilt ein Problem nun als gelöst? Laut x ist ein Problem gelöst, wenn die Suche nach der Lösung abgebrochen wird. Dabei wird die Suche durch verschiedene Abbruchkriterien geleitet:
\begin{itemize}
\item \textbf{Ziel:} Was ist der Zielzustand?
\item \textbf{Operatoren:} Welche Mittel stehen mir zur Verfügung?
\item \textbf{Beschränkungen:} Was sind die Randbedingungen?
\item \textbf{Repräsentation:} In welcher Form wird das Problem repräsentiert?
\item \textbf{Eleganz der Lösung} 
\end{itemize}

\subsection{Unterscheidung von Problemen}
Probleme unterscheiden sich hinsichtlich vieler Aspekte, die beim Problemlösen berücksichtigt werden müssen.
\begin{itemize}
\item \textbf{Klarheit:} Es wird zwischen wohl-definierte und schlecht-definierten Problemen unterschieden. Wohl-definierte Probleme kennzeichnen sich durch einen eindeutigen Ausgangs- und Zielzustand sowie klar beschriebene Operatoren.
\item \textbf{Zeitskala:} Unterscheidung zwischen kurzfristigen und langfristigen Problemen. Kurzfristige Probleme lassen sich meist schnell beheben.
\item \textbf{Zeitdruck:} Bei Zeitdruck muss eine schnelle Entscheidung getroffen werden ohne die Möglichkeit alle Lösungmöglichkeiten zu durchdenken. Ohne Zeitdruck können alle Optionen in Ruhe abgewägt werden.
\item \textbf{Geforderte kognitive Aktivität:} Wenn eine Vielzahl von Maßnahmen durchgeführt werden muss, um das Ziel zu erreichen, so ist eine hohe kognitive Aktivität gefordert.
\item \textbf{Bereiche:} Die Problemlösestrategie kann davon abhängig sein in welchem Umfeld das Problem auftritt. Probleme unterscheiden sich in ihrer Art je nach Umfeld.
\end{itemize}
Außerdem kann zwischen einfachen und komplexen Problemen unterschieden werden. Ein komplexes Problem unterscheidet sich von einem einfachen Problem in der Hinsicht, dass es mehrere unbekannte Lücken \todo{bessere Wort finden} gibt. Manche treten erste bei Bearbeitung des Problems auf. Ein komplexes Problem kennzeichnet sich durch folgende Merkmale.
\begin{itemize}
\item \textbf{Komplexität der Problemsituation:} Komplexität fordert Vereinfachung durch Reduktion auf das wesentliche \todo{Zitat??}
\item \textbf{Vernetztheit der beteiligten Variablen:} Je stärker die einzelnen Aspekte des Problems und der Lösung zusammen hängen, desto komplexer ist das Problem. Es ist wichtig die Abhängigkeiten zu kennen.
\item \textbf{Dynamik der Problemsituation:} Einerseits können durch Eingriffe in ein komplexes vernetztes System Prozess in Gang gesetzt werden, die nicht beabsichtigt waren. Andererseits wartet ein Problem nicht auf eine Entscheidung. Es ist also möglich, dass sich die Situation über die Zeit verändert.
\item \textbf{Intransparenz:} Es liegen sowohl in Hinblick auf die Zielstellung, als auch auf die Variablen nicht alle erforderlichen Informationen vor. Dadurch ist Informationsbeschaffung gefordert.
\item \textbf{Projektile:} Meistens gibt es nicht nur ein Ziel sondern mehrere Teilziele. Es ist möglich, dass nicht alle Teilziele erreicht werden können. Daher ist ein Abwägen und Balancieren der Kriterien notwendig. \todo{Optimierungskriterien}
\end{itemize}

\subsection{Arten von Problemlösern}
Nicht nur Probleme können sich unterscheiden sondern auch die auch die Art Probleme zu lösen. Es wird zwischen drei bipolaren Dimensionen unterschieden.
\begin{itemize}
\item \textbf{Veränderungsorientierung:} Beim Umgang mit Grenzen und Vorgaben wird zwischen Explorer und Developer unterschieden.
	\begin{itemize}
	\item \textbf{Explorer:} Überwindet vorgegebene Grenzen und sucht Herausforderungen.
	\item \textbf{Developer:} Liebt Pläne und Vorgaben, ist meist gut organisiert und vermeidet Risiken.
	\end{itemize}
\item \textbf{Verarbeitungsstiel:}
	\begin{itemize}
	\item \textbf{External:} Lässt Ideen durch Diskussionen mit anderen wachsen. Eine unruhige Umgebung wird nicht als störend empfunden. Außerdem handelt er, während andere noch nachdenken
	\item \textbf{Internal:} Entwickelt Ideen zunächst für sich alleine und teilt sie dann. Er bevorzugt eine ruhige Umgebung und stilles Nachdenken.
	\end{itemize}
\item \textbf{Entscheidungsfokus:}
	\begin{itemize}
	\item \textbf{People:} Der personenbezogene Entscheider betrachtet zuerst die Konsequenzen in Bezug auf Personen. Er schätzt die Harmonie.
	\item \textbf{Task:} Der aufgabenbezogene Entscheider legt Wert auf begründbare, logisch nachvollziehbare Entscheidungen.
	\end{itemize}
\end{itemize}

\subsection{Einflüsse}
Neben den äußeren Faktoren, wie die Ausgangssituation und die verfügbaren Operatoren, gibt es auch innere Faktoren, wie Motivation und Emotion, die das Problemlösen maßgeblich beeinflussen. So setzt die Motivation erst den Problemlöseprozess in Gang. \todo{noch mehr ausformulieren?}
Emotionen wirken sich auf den Ablauf des Denkens aus. So vermindern negative Emotionen die Anzahl an Selbstreflektionen beim Denken und rufen Notfallreaktionen hervor. Positive Emotionen können hingegen zu Nachlässigkeit und Oberflächlichkeit führen. Wie groß der Einfluss der Emotionen auf den Problemlöseprozess ist hängt von dem Selbstkonzept des Individuums ab. Das Selbstkonzept beschreibt die Kompetenz, die in heuristische und epistemische Kompetenz eingeteilt ist. Die heuristische Kompetenz beschreibt das Zutrauen, das jemand in seine Fähigkeiten hat mit Problemsituationen umzugehen für die es keine eindeutige Verhaltensweise gibt. Die epistemische Kompetenz zeichnet sich durch das Zutrauen, eine Situation aufgrund des vorhandenen Wissens zu bewältigen, aus. Zusammen ergibt sich daraus die aktuelle Kompetenz.

Stress, Gruppen, Sprache...

\subsection{Phasen des Problemlösens}
Der Problemlöseprozess teilt sich in fünf Phasen auf. Die erste ist die \textit{Problemidentifikation}. \glqq Ein Problem ist identifiziert, wenn man Ziele setzt und erkennt, dass ein bestimmtes Ziel nicht ohne weiteres Nachdenken erreicht werden kann.\grqq Die zweite Phase ist die \textit{Ziel- und Situationsanalyse}. Dabei muss zunächst der zu erreichende Zielzustand geklärt und die Eigenschaften und Beschränkungen erkannt werden. Anschließend ist zu klären, warum es nicht geht und was zur Verfügung steht bzw. was man gebrauchen kann. Die \textit{Planerstellung} erfolgt in Phase drei. Diese umfasst die Vorbereitung des konkreten Vorgehens mit folgenden Aspekten:
\begin{enumerate}
\item Abfolgen erkennen
\item Randbedingungen erkennen
\item Zwischenzielbildung
\item Verfügbarkeit von Alternativen
\item Angemessenheit der Auflösung
\end{enumerate}
Nach der Planerstellung folgt in Phase vier die \textit{Planausführung}. Eine wichtige Voraussetzung ist dabei die Planüberwachung und Fehlerdiagnostik.

\subsection{Störungsdiagnose}
Bei der Störungsdiagnose ist im Störfall eine effiziente Problemlösung gefragt. Das Wissen und Handeln des Individuums steht dabei im Kontext technischer Systeme. Es werden zwei Wissensarten unterschieden. \todo{überarbeiten}
\begin{itemize}
\item \textbf{Strukturwissen:} 
	\begin{itemize}
	\item \textbf{Faktenwissen:} Funktionsweise und Organisation der verschiedenen Komponenten ist bekannt. Ohne diese Wissen ist eine gezielte Systemsteuerung nicht möglich.
	\item \textbf{Displaywissen:} Kenntnis über verschiedene Bedienelement und deren Funktionsweise.
	\item \textbf{Anlagenwissen:} Kenntnis der einzelnen Komponenten des Systems.
	\end{itemize}
\end{itemize}

\section{Kollaboration}
Was heißt gut Kollaboration?

\section{Kommunikation}

\section{Assistenz}
Assistenz kann viele Facetten haben. Laut Duden bedeutet Assistenz Beistand oder Mithilfe. Das Verb assistieren wird mit den Worten \glqq jemanden nach dessen Anweisungen zur Hand gehen, bei einer Arbeit oder Tätigkeit behilflich sein\grqq erklärt.
....



\subsection{Assistenzsysteme}
Laut x kann bereits ein Schraubendreher ein Assistenzsystem sein, da dieser jemanden befähigt eine Aufgabe durchzuführen, die ohne Assistenz schwer umzusetzen ist.
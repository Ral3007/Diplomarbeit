%Beschreibung der Stand der Technik

%%%%%%%%%%%%%%%%%%%%%%%%
\chapter{Stand der Technik}
\label{sec:StandDerTechnik}

%%%%%%%%%%%%%%%%%%%%%%%%

\section{Modulare Anlagen}
Aufgrund immer kürzerer Produkteinführungszeiten werden Modularisierungskonzepte entwickelt. Die Modularisierung ermöglicht eine höhere Flexibilität und beschleunigt Konzeption, Engineering, Aufbau und Inbetriebnahme der Anlage \cite{Urbas2012}. Ein Modul ist eine geschlossene funktionale Einheit und stellt eine verfahrenstechnische Grundfunktion als Dienst der Prozessführungsebene (PFE) zur Verfügung. Die Grundfunktionalitäten der PFE müssen unterstützt werden \cite{Bernshausen2016}
\begin{itemize}
\item \textbf{Mensch-Maschine-Schnittstelle:} Übertragung der Daten zur Anzeige und Bedienung
\item \textbf{Steuern und Überwachen:} Übertragung der internen Zustände des Moduls
\end{itemize}

In der Namur-Empfehlung NE 148 \cite{NAMURArbeitskreis1.122013} ist beschrieben, welche Daten an das übergeordnete Automatisierungssystem übertragen werden und welche dem Modullieferanten zur Wartungsunterstützung zur Verfügung stehen. Die Daten für das übergeordnete Automatisierungssystemen umfassen unter anderem die Verriegelungs-, Steuerungs- und Reglungsstruktur, die Prozess- und Sollwerte sowie den Status des Moduls / der Services. Für die Wartungsunterstüzungen werden nur hersteller- und modulspezifische und keine prozessspezifischen Daten übertragen.


\section{Problemlösen}
Die Gesellschaft geht davon aus, dass Probleme selbstverständlich existieren. Probleme entstehen allerdings erst, wenn eine konkrete Zielsetzung vorhanden ist, die sich nicht durch Routine erreichen lässt. Ohne Handlungsziele gäbe es keine Probleme. \cite{Funke2015, Betsch2011,  Dorner1984}

Liegt ein Problem vor so könnte der Problemlöseprozess sehr einfach sein, indem der Ausgangszustand erkannt, der Zielzustand festgelegt und die Operatoren gefunden werden. Allerdings haben alle diese Aspekte Eigenschaften, die den Prozess erschweren. So kann der Ausgangszustand nicht immer klar definiert sein und es muss eindeutig sein, welche Voraussetzungen als erfüllt angenommen werden können. Bei einem unklaren Ausgangszustand lässt sich auch der Zielzustand nicht eindeutig beschreiben. Bei Betrachtung der Operatoren, die notwendig sind, um einen Ausgangszustand in einen Zielzustand zu überführen, fällt auf, dass diese mit dem Ziel zusammen hängen. Entweder wird der Zielzustand betrachtet und nach geeigneten Operatoren gesucht, die unter Umständen nicht vorhanden sind. Oder es sind bestimmte Operatoren vorhanden und es wird davon ausgehend das bestmöglichste Ziel bestimmt. \cite{Funke2015}

Wann gilt ein Problem nun als gelöst? Laut x ist ein Problem gelöst, wenn die Suche nach der Lösung abgebrochen wird. Dabei wird die Suche durch verschiedene Abbruchkriterien geleitet \cite{Funke2015}:
\begin{itemize}
\item \textbf{Ziel:} Was ist der Zielzustand?
\item \textbf{Operatoren:} Welche Mittel stehen mir zur Verfügung?
\item \textbf{Beschränkungen:} Was sind die Randbedingungen?
\item \textbf{Repräsentation:} In welcher Form wird das Problem repräsentiert?
\item \textbf{Eleganz der Lösung} 
\end{itemize}

\subsection{Unterscheidung von Problemen}
Probleme unterscheiden sich hinsichtlich vieler Aspekte, die beim Problemlösen berücksichtigt werden müssen. \cite{Betsch2011}
\begin{itemize}
\item \textbf{Klarheit:} Es wird zwischen wohl-definierte und schlecht-definierten Problemen unterschieden. Wohl-definierte Probleme kennzeichnen sich durch einen eindeutigen Ausgangs- und Zielzustand sowie klar beschriebene Operatoren.
\item \textbf{Zeitskala:} Unterscheidung zwischen kurzfristigen und langfristigen Problemen. Kurzfristige Probleme lassen sich meist schnell beheben.
\item \textbf{Zeitdruck:} Bei Zeitdruck muss eine schnelle Entscheidung getroffen werden ohne die Möglichkeit alle Lösungmöglichkeiten zu durchdenken. Ohne Zeitdruck können alle Optionen in Ruhe abgewägt werden.
\item \textbf{Geforderte kognitive Aktivität:} Wenn eine Vielzahl von Maßnahmen durchgeführt werden muss, um das Ziel zu erreichen, so ist eine hohe kognitive Aktivität gefordert.
\item \textbf{Bereiche:} Die Problemlösestrategie kann davon abhängig sein in welchem Umfeld das Problem auftritt. Probleme unterscheiden sich in ihrer Art je nach Umfeld.
\end{itemize}
Außerdem kann zwischen einfachen und komplexen Problemen unterschieden werden. Ein komplexes Problem unterscheidet sich von einem einfachen Problem in der Hinsicht, dass es mehrere unbekannte Lücken \todo{bessere Wort finden} gibt \cite{Betsch2011}. Manche treten erst bei Bearbeitung des Problems auf. Ein komplexes Problem kennzeichnet sich durch folgende Merkmale. \cite{Funke2015}
\begin{itemize}
\item \textbf{Komplexität der Problemsituation:} Komplexität fordert Vereinfachung durch Reduktion auf das wesentliche \todo{Zitat??}
\item \textbf{Vernetztheit der beteiligten Variablen:} Je stärker die einzelnen Aspekte des Problems und der Lösung zusammen hängen, desto komplexer ist das Problem. Es ist wichtig die Abhängigkeiten zu kennen.
\item \textbf{Dynamik der Problemsituation:} Einerseits können durch Eingriffe in ein komplexes vernetztes System Prozess in Gang gesetzt werden, die nicht beabsichtigt waren. Andererseits wartet ein Problem nicht auf eine Entscheidung. Es ist also möglich, dass sich die Situation über die Zeit verändert.
\item \textbf{Intransparenz:} Es liegen sowohl in Hinblick auf die Zielstellung, als auch auf die Variablen nicht alle erforderlichen Informationen vor. Dadurch ist Informationsbeschaffung gefordert.
\item \textbf{Projektile:} Meistens gibt es nicht nur ein Ziel sondern mehrere Teilziele. Es ist möglich, dass nicht alle Teilziele erreicht werden können. Daher ist ein Abwägen und Balancieren der Kriterien notwendig. \todo{Optimierungskriterien}
\end{itemize}

\subsection{Arten von Problemlösern}
Nicht nur Probleme können sich unterscheiden sondern auch die auch die Art Probleme zu lösen. Es wird zwischen drei bipolaren Dimensionen unterschieden. \cite{Betsch2011}
\begin{itemize}
\item \textbf{Veränderungsorientierung:} Beim Umgang mit Grenzen und Vorgaben wird zwischen Explorer und Developer unterschieden.
	\begin{itemize}
	\item \textbf{Explorer:} Überwindet vorgegebene Grenzen und sucht Herausforderungen.
	\item \textbf{Developer:} Liebt Pläne und Vorgaben, ist meist gut organisiert und vermeidet Risiken.
	\end{itemize}
\item \textbf{Verarbeitungsstiel:}
	\begin{itemize}
	\item \textbf{External:} Lässt Ideen durch Diskussionen mit anderen wachsen. Eine unruhige Umgebung wird nicht als störend empfunden. Außerdem handelt er, während andere noch nachdenken
	\item \textbf{Internal:} Entwickelt Ideen zunächst für sich alleine und teilt sie dann. Er bevorzugt eine ruhige Umgebung und stilles Nachdenken.
	\end{itemize}
\item \textbf{Entscheidungsfokus:}
	\begin{itemize}
	\item \textbf{People:} Der personenbezogene Entscheider betrachtet zuerst die Konsequenzen in Bezug auf Personen. Er schätzt die Harmonie.
	\item \textbf{Task:} Der aufgabenbezogene Entscheider legt Wert auf begründbare, logisch nachvollziehbare Entscheidungen.
	\end{itemize}
\end{itemize}

\subsection{Einflüsse}
Neben den äußeren Faktoren, wie die Ausgangssituation und die verfügbaren Operatoren, gibt es auch innere Faktoren, wie Motivation und Emotion, die das Problemlösen maßgeblich beeinflussen. So setzt die Motivation erst den Problemlöseprozess in Gang. \todo{noch mehr ausformulieren?} \cite{Dorner1984}

Emotionen wirken sich auf den Ablauf des Denkens aus. So vermindern negative Emotionen die Anzahl an Selbstreflektionen beim Denken und rufen Notfallreaktionen hervor. Positive Emotionen können hingegen zu Nachlässigkeit und Oberflächlichkeit führen. Wie groß der Einfluss der Emotionen auf den Problemlöseprozess ist hängt von dem Selbstkonzept des Individuums ab. Das Selbstkonzept beschreibt die Kompetenz, die in heuristische und epistemische Kompetenz eingeteilt ist. Die heuristische Kompetenz beschreibt das Zutrauen, das jemand in seine Fähigkeiten hat mit Problemsituationen umzugehen für die es keine eindeutige Verhaltensweise gibt. Die epistemische Kompetenz zeichnet sich durch das Zutrauen, eine Situation aufgrund des vorhandenen Wissens zu bewältigen, aus. Zusammen ergibt sich daraus die aktuelle Kompetenz. \cite{Dorner1984}

Stress, Gruppen, Sprache...

\subsection{Phasen des Problemlösens}
Der Problemlöseprozess teilt sich in fünf Phasen auf. Die erste ist die \textit{Problemidentifikation}. \glqq Ein Problem ist identifiziert, wenn man Ziele setzt und erkennt, dass ein bestimmtes Ziel nicht ohne weiteres Nachdenken erreicht werden kann.\grqq Die zweite Phase ist die \textit{Ziel- und Situationsanalyse}. Dabei muss zunächst der zu erreichende Zielzustand geklärt und die Eigenschaften und Beschränkungen erkannt werden. Anschließend ist zu klären, warum es nicht geht und was zur Verfügung steht bzw. was man gebrauchen kann. Die \textit{Planerstellung} erfolgt in Phase drei. Diese umfasst die Vorbereitung des konkreten Vorgehens mit folgenden Aspekten:
\begin{enumerate}
\item Abfolgen erkennen
\item Randbedingungen erkennen
\item Zwischenzielbildung
\item Verfügbarkeit von Alternativen
\item Angemessenheit der Auflösung
\end{enumerate}
Nach der Planerstellung folgt in Phase vier die \textit{Planausführung}. Eine wichtige Voraussetzung ist dabei die Planüberwachung und Fehlerdiagnostik. \todo{hört aprupt auf} \cite{Betsch2011}

\subsection{Störungsdiagnose}
Bei der Störungsdiagnose ist im Störfall eine effiziente Problemlösung gefragt. Das Wissen und Handeln des Individuums steht dabei im Kontext technischer Systeme. Es werden zwei Wissensarten unterschieden. \todo{überarbeiten}
\begin{itemize}
\item \textbf{Strukturwissen:} 
	\begin{itemize}
	\item \textbf{Faktenwissen:} Funktionsweise und Organisation der verschiedenen Komponenten ist bekannt. Ohne diese Wissen ist eine gezielte Systemsteuerung nicht möglich.
	\item \textbf{Displaywissen:} Kenntnis über verschiedene Bedienelement und deren Funktionsweise.
	\item \textbf{Anlagenwissen:} Kenntnis der einzelnen Komponenten des Systems.
	\end{itemize}
\end{itemize}

\section{Kollaboration}
Kollaboration bietet die Chance verteilte Informationen für das Lösen von Problemen zu nutzen und unterscheidet sich von reiner Kooperation.  Mit Kooperation ist eine Arbeitsteilung gemeint, bei der jede Person eine konkrete Aufgabe zugeteilt wird und die Ergebnisse zum Schluss zusammen getragen werden. Kollaboration ist durch Symmetrie von Wissen, einem gemeinsamen Ziel und der Zusammenarbeit gekennzeichnet. Dabei spielt insbesondere die Interaktion untereinander eine große Rolle, da diese den kollaborativen Lerneffekt fördert. 

\section{Kommunikation}
Kommunikation ist beim kollaborativen Problemlösen ein wichtiger Aspekt. Mittels Kommunikation kann das gemeinsame Verständnis des Problems hergestellt und aufrecht gehalten werden. Um Missverständnisse vorzubeugen ist es erforderlich klar und geeignete Fragen zu stellen. Das stellen von Fragen ist wichtig, um ungeteilte Informationen auszutauschen. Ebenso wichtig ist das richtige zuhören, da die meisten besser Informationen geben als aufnehmen können. Zudem sollten nur die Informationen weiter gegeben werden, die für die Situation notwendig sind.

Während Menschen direkt kommunizieren können ist bei der Kommunikation mit einer digitalen Assistenz noch ein zusätzliches System notwendig. Welche Systeme dafür verwendet werden können ist in Abschnitt xx näher beschrieben. Bei Betrachtung des Aspekts WIE kommuniziert wird, fällt auf, dass es vielfältige Möglichkeiten gibt. Stellt das Interaktionssystem beispielsweise nur eine Kommandozeile zur Kommunikation zur Verfügung so muss der Nutzer mit Kommandos kommunizieren. Andersrum kann ein stummer Mensch schwierig mit einem System interagieren, das nur gesprochene Sprache verarbeiten kann. Es ist also notwendig, dass geeignete Mittel zur Kommunikation zur Verfügung stehen cite.

\section{Assistenz}
Laut Duden bedeutet Assistenz Beistand oder Mithilfe. Das Verb assistieren wird mit den Worten \glqq jemanden nach dessen Anweisungen zur Hand gehen, bei einer Arbeit oder Tätigkeit behilflich sein\grqq \ beschrieben. In der Literatur finden sich eine Vielzahl von Definitionen. Diese reichen vom Schraubendreher, über autonome Ausführung von Funktionen bis hin zur individualisierten Nutzerunterstützung. In x wird Assistenz mehrere Stufen anhand des Autonomiegrades eingeteilt(, an denen sich diese Arbeit im weiteren Verlauf orientiert):
\begin{enumerate}
\item \textbf{Automatisches Ausführen von Funktionen:} Die Funktion wird nicht durch den Benutzer ausgeführt. Ein Beispiel ist das automatische Herunterfahren eine Anlage bei gravierenden Störungen.
\item \textbf{Unterstützung bei einem vorab definierten Anwendungsfall:}
\item \textbf{Erkennung der Intention des Nutzers und Vorschlag von geeigneten Schritten}
\end{enumerate}

\subsection{Anforderungen an digitale Assistenzsysteme}
Die Assistenz soll den Menschen ideal unterstützen. Dabei sind seine Fähigkeiten zu berücksichtigen und eine Überlastung ist zu vermeiden. Es müssen dabei folgende Aspekte beachtet werden:
\begin{itemize}
\item \textbf{Interaktivität:} Dem Mensch muss die Möglichkeit zur Interaktion gegeben werden. Die Ziele und Aufgaben sollten formulierbar sein ohne Rücksicht auf das System nehmen zu nehmen.
\item \textbf{Diagnose:} Das Assistenzsystem muss wissen, welche Effekte auftreten können, wenn der Nutzer fehlerhafte Eingaben tätigt.
\item \textbf{Korrektur:} Wenn die Handlung des Nutzers von den Anweisungen abweicht, so muss das Assistenzsysteme diese Handlung trotzdem unterstützen können.
\end{itemize}

\subsection{Einsatz von digitaler Assistenz}
Digitale Assistenz findet sich mittlerweile überall. So gibt es für fast jede Anwendung eine Onlinehilfe, die mit Tool-Tipps Assistenz leistet. Im Alltag finden sich für blinde Menschen akustische Signale an Ampeln wieder. Zuhause gibt es mittlerweile Smart Home Geräte, die automatisch die Heizung ausstellen, wenn das Fenster geöffnet wird. Das Handy fügt automatisch Termine aus eMails dem Terminkalender hinzu und erinnert anschließend an den Termin.

Im Prozessumfeld kann digitale Assistenz folgende Aufgaben haben:
\begin{itemize}
\item \textbf{Aktivierung}
	\begin{itemize}
	\item \textbf{Warnung:} Die Assistenz warnt bevor der Mensch eventuell einen Fehler macht.
	\item \textbf{Signale:} Die Assistenz sorgt dafür, dass alle relevanten Informationen für den Nutzer erkennbar sind.
	\end{itemize}
\item \textbf{Informationsintegration:} Darstellung von Symbolen, die dem Nutzer bekannt sind (z.B. km/h vs. mph). Erläuterung von möglichen Konsequenzen.
	\begin{itemize}
	\item \textbf{Kennzeichnung:} Legenden für die verschiedenen Symbole.
	\item \textbf{Erklärung:} Mit Sicht auf die Interessen und das Wissen des Nutzers.
	\end{itemize}
\item \textbf{Entscheidungen:} Unterstützung bei der Auswahl, was als nächstes getan werden muss
	\begin{itemize}
	\item \textbf{Supply:} Darstellung aller möglichen Informationen und Optionen
	\item \textbf{Filter:} Es werden nur die Informationen und Optionen dargestellt, die für die Aufgabe wichtig sind.
	\item \textbf{Berater:} Die Assistenz liefert einen Vorschlag. Der Mensch kann entscheiden, ob er die vorgeschlagene Option durchführt.
	\end{itemize}
\end{itemize}

\subsection{Assistenzsysteme}
Zur Verbindung der Fähigkeiten des menschlichen Nutzers und den Systemfunktionen können Assistenzsysteme verwendet werden. Ein Assistenzsysteme besteht aus mehreren Komponenten. Der Eingabemethode, also die Art und Weise, wie der Mensch mit dem System interagieren kann. Der tatsächlichen Schnittstelle, beispielsweise einem Computer, der Informationen anzeigt. Und dem Assistent als solches, der Informationen verknüpft und bereit stellt.

\subsubsection*{Eingabemethoden}
Die Eingabemethoden orientieren sich maßgeblich an den Fähigkeiten des Menschen. Der Mensch verwendet meist Hände und Sprache. Insbesondere die Hände bieten eine Vielzahl an Möglichkeiten mit einem System zu interagieren. So können Hilfsmittel, wie Maus oder Tasten, verwendet werden oder die Interaktion erfolgt direkt mit Gesten oder über ein Touchscreen. Die Interaktionsmöglichkeiten sind in Tabelle \ref{tab:Interaktionsmöglichkeit} aufgeführt.

\begin{sidewaystable}[ph!]
\begin{tabular}{p{3cm}|p{4cm}|p{4cm}|p{3cm}|p{3cm}}
	\textbf{Interaktions-möglichkeit} & \textbf{Merkmale} & \textbf{Vorteile} & \textbf{Nachteile} & \textbf{Einsatz} \\
	\hline
	Taster & führt zugewiesene Aktions aus & & & \\
	\hline
	Maus & zweidimensionale Bewegung & Bedienung ist einfach zu erlernen & benötigt eine ebene saubere Fläche & vor allem im Bürobereich\\
	\hline
	Joystick & wird durch kippen bedient & schnelle Richtungswechsel möglich & & als Mausersatz, bei Zielverfolgungsaufgaben \\
	\hline
	Touchscreen & Interaktion durch Berühren des Bildschirms & Direkte Bedienung, keine zusätzliche Hardware nötig & Verschmutzt schnell & weitreichend: von Industrie bis Labor \\
	\hline
	Spracheingabe & sichere Erkennung muss gewährleistet sein & einfach zu bedienen & & Auswahlvorgänge, Kommandos \\
	\hline
	Gesten & werden mit Kamera erfasst & & & \\
	\hline
\end{tabular}
\label{tab:Interaktionsmöglichkeit}
\caption{Interaktionsmöglichkeiten mit einem Assistenzsystem}
\end{sidewaystable}

\subsubsection*{Schnittstelle}
Mit Schnittstelle ist gemeint, wie und mit welchen Mitteln Informationen dem Nutzer zur Verfügung gestellt werden. Die Mittel sind in Tabelle \ref{tab:Interaktionssystem} dargestellt.

\begin{sidewaystable}[ph!]
\begin{tabular}{p{2,6cm}|p{3,5cm}|p{3,5cm}|p{3,5cm}|p{3,5cm}}
	\textbf{Interaktions-system} & \textbf{Funktionsweise} & \textbf{Vorteile} & \textbf{Nachteile} & \textbf{Anwendung} \\
	\hline
	Projektor & Beleuchtung des relevanten Objekts & gut geeignet für Arbeiter mit kognitiven Einschränkungen & Einsatz ist abhängig von geforderter Projektionsgenauigkeit & Unterstützung des Kommissionierungsvorgangs, Bohrlöcher \\
	\hline
	AR-Brillen & Einblendung von Zusatzinformationen in das Sichtfeld & handfree, komplexe Arbeitsabläufe können fehlerärmer umgesetzt werden & Sichtfeld ist geringfügig eingeschränkt & Checklisten, Anleitungen, Anzeige von Messdaten \\
	\hline
	Headset & gibt akustisch Hinweise und Informationen & handsfree, Verwendbar, wenn visueller Kanal nicht zur Verfügung steht & funktioniert nur bedingt in lauter Umgebung & Call-Center, Logistik \\
	\hline
	Smartwatch & kann wenige wichtige Informationen anzeigen & handsfree, kompakt & begrenzte Displaygröße & Navigation, Information \\
	\hline
	Tablet & & einfache Handhabung & nur eine freie Hand & Anleitung, Wartung von Maschinen\\
	\hline
	stationärer Computer & & großer Bildschirm & nicht transportabel & \\
\end{tabular}
\label{tab:Interaktionssystem}
\caption{Interaktionssysteme zur Bereitstellung von Informationen}
\end{sidewaystable}

\section{Gestaltung von Mensch-Maschine-Schnittstellen}
Die richtige Gestaltung der Mensch-Maschine-Schnittstelle ist essenziell. Durch die steigende Komplexität von Maschinen und Anlagen wird meist auch die Bedienung komplexer. Umso wichtiger ist eine nutzerfreundliche Gestaltung. Diese orientiert sich maßgeblich an den Bedürfnissen des Nutzers, welche in den Entwicklungsprozess mit einzubeziehen sind. Es gibt eine Vielzahl an Richtlinien, die erläutern, was eine ergonomisch gute Gestaltung von Benutzerschnittstellen ausmacht. Die gute Gestaltung soll Benutzungsprobleme vermeiden.

\subsection{Ergonomisch gute Gestaltung}
In der DIN EN ISO 9241 sind Empfehlungen für die Ergonomie der Mensch-System-Interaktion aufgelistet. An dieser Stelle wird nur auf einige für diese Arbeit relevante Aspekte eingegangen. So sind in Teil 110 die Grundsätze der Dialoggestaltung beschrieben:
\begin{itemize}
\item \textbf{Aufgabenangemessenheit:} Funktionalität und Dialog sollen den Eigenschaften der Arbeitsaufgabe entsprechen.
\item \textbf{Selbsbeschreibungsfähigkeit:}  Es muss eindeutig sein, an welcher Stelle sich der Nutzer befindet, welche Handlungen durchgeführt werden können und wie diese auszuführen sind.
\item \textbf{Erwartungskonformität:} Der Dialog entspricht den anerkannten Konventionen und ist vorhersehbar.
\item \textbf{Lernförderlichkeit:} Der Nutzer wird beim Erlernen der Nutzung des interaktiven Systems unterstützt.
\item \textbf{Steuerbarkeit:} Der Nutzer hat Einfluss auf Richtung und Geschwindigkeit des interaktiven Systems.
\item \textbf{Fehlertoleranz:} Das beabsichtigte Arbeitsergebnis kann bei fehlerhaften Eingaben trotzdem mit keinem oder minimalem Korrekturaufwand erreicht werden.
\item \textbf{Individualisierbarkeit:} Nutzer kann die Darstellung von Informationen so ändern, dass sie seinen Bedürfnissen und Fähigkeiten entsprechen.
\end{itemize}
Es ist deutlich zu erwähnen, dass in den meisten Fällen nicht alle Aspekte gleichermaßen berücksichtigt werden können.

\subsection{Individualisierung}


Für die ergonomische gute Gestaltung von Benutzerschnittstellen gibt es eine Vielzahl von Richtlinien.




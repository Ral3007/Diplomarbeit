%Beschreibung der Stand der Technik

%%%%%%%%%%%%%%%%%%%%%%%%
\chapter{Stand der Technik}
\label{sec:StandDerTechnik}

%%%%%%%%%%%%%%%%%%%%%%%%

\section{Modulare Anlagen}
Aufgrund immer kürzerer Produkteinführungszeiten werden Modularisierungskonzepte entwickelt. Die Modularisierung ermöglicht eine höhere Flexibilität und beschleunigt Konzeption, Engineering, Aufbau und Inbetriebnahme der Anlage \cite{Urbas2012}. Ein Modul ist eine geschlossene funktionale Einheit und stellt eine verfahrenstechnische Grundfunktion als Dienst der Prozessführungsebene (PFE) zur Verfügung. Die Grundfunktionalitäten der PFE müssen unterstützt werden \cite{Bernshausen2016}
\begin{itemize}
\item \textbf{Mensch-Maschine-Schnittstelle:} Übertragung der Daten zur Anzeige und Bedienung
\item \textbf{Steuern und Überwachen:} Übertragung der internen Zustände des Moduls
\end{itemize}

In der Namur-Empfehlung NE 148 \cite{NAMURArbeitskreis1.122013} ist beschrieben, welche Daten an das übergeordnete Automatisierungssystem übertragen werden und welche dem Modullieferanten zur Wartungsunterstützung zur Verfügung stehen. Die Daten für das übergeordnete Automatisierungssystemen umfassen unter anderem die Verriegelungs-, Steuerungs- und Reglungsstruktur, die Prozess- und Sollwerte sowie den Status des Moduls / der Services. Für die Wartungsunterstüzungen werden nur hersteller- und modulspezifische und keine prozessspezifischen Daten übertragen.


\section{Problemlösen}

\section{Kollaboration}
Was heißt gut Kollaboration?



\section{Assistenz}
Assistenz kann viele Facetten haben. Laut Duden bedeutet Assistenz Beistand oder Mithilfe. Das Verb assistieren wird mit den Worten \glqq jemanden nach dessen Anweisungen zur Hand gehen, bei einer Arbeit oder Tätigkeit behilflich sein\grqq erklärt.
....



\subsection{Assistenzsysteme}
Laut x kann bereits ein Schraubendreher ein Assistenzsystem sein, da dieser jemanden befähigt eine Aufgabe durchzuführen, die ohne Assistenz schwer umzusetzen ist.
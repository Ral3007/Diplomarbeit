\documentclass{ifaPoster}

\usepackage[utf8]{inputenc}
\usepackage[ngerman]{babel}
\usepackage{blindtext}
\usepackage{amsmath} % für multline
\usepackage{subfig} % für subfloat
\usepackage{todonotes}

\ifaAuthor{Meret Feldkemper}
\ifaTitle{Kollaborative Problemlösung in modularen Anlagen mittels persönlicher digitaler Assistenz}
\ifaSupervisorA{Dipl.-Ing. Sebastian Heinze}
%\ifaSupervisorB{Dipl.-Ing. Betreuer B}
%\ifaSupervisorC{Dipl.-Ing. Betreuer C}
\ifaProfessor{Prof. Dr.-Ing. habl. Leon Urbas}
\ifaDayOfSubmission{02.05.2019}
\ifaThesis{Diplomarbeit}
\ifaPhoto{Passbild_Platzhalter.pdf}
 
\begin{document}

\section{Motivation und Aufgabenstellung}


\section{Konzept}


\section{Zusammenfassung}


 {\tiny\renewcommand{\section}[2]{}%
 	 \begin{thebibliography}{8.5}
 	 \bibitem{hahn_virtuelle_2016}
		Anna Hahn, Stefan Pech und Leon Urbas. {\glqq Virtuelle funktionale Module in der Prozessindustrie\grqq}. {In: \textit{Atp Edition}} 58.11 (2016), S. 54-63.
		\bibitem{klose_erstellung_2018}
		Anselm Klose: {\glqq Erstellung von Pfadregeln basierend auf einer qualitativen Bewertung von Messstellen einer Bestandanlage\grqq}. Diplomarbeit. Dresden: Technische Universität Dresden, 2018.
		\bibitem{mehlig_erweiterung_2018}
		Sebastian Mehlig: {\glqq Erweiterung einer Bewertungsmethode von Messstellen innerhalb einer verfahrenstechnischen Prozessanlage um quantitative Regeln\grqq}. Diplomarbeit. Dresden: Technische Universität Dresden, 2018.
	\end{thebibliography}}
	
\end{document}